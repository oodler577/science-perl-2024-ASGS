\documentclass{article}
\bibliographystyle{plain}

% Language setting
% Replace `english' with e.g. `spanish' to change the document language
\usepackage[english]{babel}

% Set page size and margins
% Replace `letterpaper' with `a4paper' for UK/EU standard size
\usepackage[letterpaper,top=2cm,bottom=2cm,left=3cm,right=3cm,marginparwidth=1.75cm]{geometry}

% Useful packages
\usepackage{amsmath}
\usepackage{graphicx}
\usepackage[colorlinks=true, allcolors=blue]{hyperref}

\usepackage{listings,xcolor}

\lstdefinestyle{perl}
{
   language=Perl,
   basicstyle=\ttfamily,
   keywordstyle=\bfseries\color{green!40!black},
   commentstyle=\color{purple!40!black},
   identifierstyle=\color{blue},
   stringstyle=\color{orange},
   alsoletter={\%},
}
\lstset{style=perl}

\title{Building A Storm Surge Forecasting System That Saves Lives}

\author{
Brett Estrade \\
\large\textit{brett.estrade@coastalcomputingservices.net}
\\
Jason G. Fleming, PhD. \\
\large\textit{jason.fleming@seahorsecoast.com}
}

\begin{document}
\maketitle

\begin{abstract}
The ADCIRC Surge Guidance System is a portable real-time operational storm surge
fore- casting framework, forged over 15 Atlantic hurricane seasons, that is used
to deliver critical information to emergency managers on federal, state, and
local levels in Louisiana, Texas, and North Carolina and within groups such
FEMA, NOAA, and DHS. Over the years it has saved millions of dollars in time,
property, and emergency assets. It has also likely saved many lives. This paper
discusses its humble origins from a collaboration between LSU and UNC in the
early wake of Hurricane Katrina (2005), to its first real test during Hurricane
Gustav (2008) and successes over the years. Much of this paper discusses the
technical aspects of the system, how the user experience has been tailored for
real-time operations, and the technical decisions that have been made leading
directly to its success as a robust and adaptable framework. Most relevant ’for
the intended audience is how using Perl, Bash shell scripting, and standard Unix
tools has made all of this possible.
\end{abstract}

\section{Introduction}

It is hard to believe that in 2001 when weather forecasting models were well
know and used across society, that the the idea of providing such forecasting
litoral, or nearshore zones was the realm of specialized oceangraphers and the
military. But it is true, that until the infamous Hurricane Katrina in 2005 that
that nobody outside this small cadre of specialists ran these kinds of models or
cared much.  As it came to pass, Katrina exposed the crucial need for providing
autmated storm surge for- casting. And wider society was reminded that it is the
water that kills the most people during hurricanes, not the wind. Far more than
a slow moving, foreboding and wet tornado, a hurricane is a wholly different
kind of threat. Along with its many inches of rain and harsh winds, can come a
wall of that may in some places exceed 30 feet. And this was the case with
Katrina.  The mathematics behind such modelling capabilities had existed for a
long time \cite{luettich1992adcirc}, and in fact was pioneered by the Dutch; who
themselves had been victims of grave flooding events over the centuries. It was
not until the late 1980s that a model was developed that would revolutionize the
ability to accurately model the way that water behaved near the shorelines.

\section{ADCIRC - The Advanced Circulation Model}

The ADCIRC model is the main kernel around which the forecasting system ins
build. It automates all aspects of setting up the model, running it (usually in
on a parallel Linux cluster), processing the results, and send the results to a
geographical information system for viewing on the web.  The computational model
itself is internationally recognized as the best way model the the behavior of
large mass of water as it is piled up against a complex coastline geometry. It
is used to study the impacts of past hurricanes in the Gulf of Mexico and the
Atlantic coastline of the United States. It is to verify insurance claims after
these sort of tropical surge events. It is even used to forecast potential
impacts of hypothetical and real time tropical events.  The application area
discussed here is that of forecasting potential impacts of a tropical event as
it heads towards making landfall somewhere in the United States. The system is
essentially the automation of running the model, just as someone would run it
manually. Only here the system automates all aspects of the set up, execution,
and post processing of results into products and forms that are readily consumed
by the interested parties, be they the general public or Federal and State level
emergency managers.  Recently, after many years in a state of ambiguity, the
source code for ADCIRC was released as open source software under the GNU GPL
3.0 license. This has opened up opportunities for professional ADCIRC support
companies such as ADCIRC Live \footnote{\url{https://adcirc.live}} to offer support
packages and SaaS platforms for helping those using ADCIRC professionally or as
a public service.

\section{ADCIRC Input Data}

ADCIRC has a minimal set of input files that it needs to run. It needs
unstructured finite element mesh that describes the physical geographical domain
(e.g., the coastline and nearshore features such as levees, piers, etc) and a
settings file. The mesh is referred to colloquially as the "fort.14" file; the
settings file is referred to as the "fort.15" file. ADCIRC may be run with no
forcing turned on, but without additional input files may be run with tidal
standard tidal forces being applied. Most usefully, ADCIRC is run with winds
that are driving the surface of the water domain. The file that contains the
winds, or meteorological forcing, is referred to as the "fort.22" file. This
file can take on different types of formats.

One someone is setting up the model for the first time, the next big challenge
is generating relevant fort.22 files for the mesh. Most do not start out
creating their own mesh, this is considered an advanced activity. Most users
start with a well established unstructured mesh, then work to convert some
meteorological data into this fort.22. For example, if someone wanted to run
ADCIRC using a particular mesh and the wind data observed by a recent hurricane

\subsection{The Unstructured Mesh}

The ADCIRC model is centered around the concept of a physical domain, i.e., the
coastline along a large body of water capable of generated a surge of water
resulting from meteorological forcing - i.e., wind. When the model was
originally developed, the decision was made that the physical domain would be
represented with what is called finite elements.  The main reason was that in
order to model complex coastline geometries, the representation used needed to
allow for varying degrees of resolution such that there were many points of
interest near the shore and relatively few points of interest in the open ocean.
Any number of polygonal shapes could be used, but ADCIRC uses triangular
elements. So a mesh consists of points in space at specific locations and a list
of neighboring nodes to which it is connected.

This unstructured, or finite element mesh, is as one might expect defined as a
graph would be. I.e., the mesh file, or fort.14, is simply a list of all the
points in the domain; plus a list of all edges connecting different points.
ADCIRC is strictly 2 dimensional internally, but because the depth of the water
affects the surge, each mesh does also require the bathymetric depth at each
point. This information is also contained in the fort.14 file.  Once selected
for a model run, this file does not change and is considered static in any
consideration of automation or continuous running scenarios.

\subsection{Settings File}

The settings files controls many of the runtime settings of the ADCIRC model.
The most important as they relate to the automation of the model are that the
specify when in the "model time" output is produced. Output includes the actual
prediction values of height of the water and 2 dimensional water velocities, but
it also includes the outputting of a state or "hot start" file. This state file
fundamental to any system automation of the model that is meant to run a
simulation continuously. More on this is described later in this paper in the
section regarding considerations for running ADCIRC continuously. Another
important aspects of this file that are important include the number of days of
the simulation (in model time).  When considering the automation of ADCIRC model
runs and the continuous running across successive restarts, starting with the
previous last mode state; the fort.15 must be updated for each successive start.

\subsection{Meteorological Forcing}

North American Meteorological Forecasts (NAM) ASGS may be run in a daily
forecasting mode where it watches for available NAM forecast data. This data is
typically made available 4 times in a 24 hour period at: 00Z, 06Z, 12Z, and 18Z.
ASGS uses a Perl script to create a meteorological wind forcing file, commonly
referred to as a fort.22. The data conversion process involves interpolated the
NAM gridded data values onto the unstructured nodes of the unstructured mesh
that ADCIRC uses to represent complex coastline geometries.  NAM data is also
used when the NHC identifies areas of interest, but are not yet well formed
enough to begin issue advisories. ASGS is able to shift from NAM mode to NHC
mode once official advisories begin. These advisories are short text file and do
not contain explicit forecast data, so here again Perl is used.  Updated input
files are fetched and prepared between a previous and success model run. In the
case of hurricane forecasting, much of this meteorological data is a speculative
forecast. One of it is, however, based on actual observations; this must be used
in order to catch the model up to the present time with real data so that it
reflects what has actually happened. In the literature, this is referred to as a
\textit{hindcast}.  National Hurricane Center Forecasts

The advisories that are issued but the NHC contain forecasting about the future
projected path of the center of pressure of the system. In addition to this, it
contains a set of date that describes the forasted path of the system and lines
of similar pressures (VERIFY).

A Perl script parses this information out of the NHC forecast advisory, and
converts it into a format that may be used by ADCIRC to generate a wind field
gradient. This wind field is then mapped internally inside of ADCIRC to the
mesh, and used to drive the model state as if it were provided explicit wind
information. The internal wind generation is done using what’s called the
Generalized Asymmetric Holland Model, or just GAHM.

\subsection{Other Input Files}

For the sake of completeness, it should be mentioned that ADCIRC contains other
input files that describe aspects such as nodal attributes, which provides
additional values mapped to each node (the fort.13 file). One example is surface
roughness, which is used to specify the wind damping effect of bare dry land,
forested areas, swamps and marshes, etc. These attributes have a very important
effect to how the water behaves once dry areas have become wet or hurricanes
begin to effect areas buffered by wetlands and marshes. The latter many coastal
communities in Louisiana have traditionally been spared from storm surge during
hurricanes is that they sit behind wetlands; these tend to have a protective
effect from the surge. We they are depleted by manmade or natural causes, this
protection goes away because less of the energy that contributes to the storm
surge is dissipated.

\section{Automating ADCIRC}

ADCIRC lends itself exceptionally well to automation on Unix and Unix-like
operating systems with familiar approaches such as shell and Perl scripting,
especially in large timesharing High Performance Computing environments. Since
it is a rare individual that possess acumen in both the science and in
proficient system scripting and software engineers, environments where ADCIRC
automation is needed requires individuals with these complementary skill sets.
Generally speaking, the the automation of ADCIRC actually requires addressing
many areas. For a "system" to claim that it fully automates ADCIRC, the system
must address all of the following capabilities in some manner:

Missing from this list of capabilities are two topics. One has to do with the
coupling of other models that specialize in different physical processes. The
has to do with data assimilation or adjusting of the model output data based on
some external method of analysis.

\subsection{Building ADCIRC}

The successful building of ADCIRC on a computing platform one has at their
disposal is considered a rite of passage in the ADCIRC Community. Many graduate
students, engineering consultants, and computer support staff members will face
this challenge very soon after the decision is made to use ADCIRC in some way.
The activity of building ADCIRC on a new platform is referred to as "porting".
The ADCIRC source code ships with both its traditional Makefile-based \cite{}
building systems and a newer, but not necessarily cooler CMake \cite{}. Most
users prefer the traditional Makefile, since it is more amenable to the process
of tinkering until it works. Most porting efforts require figuring out the
work/cmplrflags.mk file. On the rare occasion, often when it involves a new
compiler on a new system, there might need to be minor updating of the Fortran
90 code. Due to this reality, the Fortran 90 code itself has accumulated CPP
style IFDEF sections.  Recently, the building and deployment of ADCIRC has been
made much more accessible on the systems that are actively supported by ASGS
\cite{}. Other systems for providing ADCIRC do exist, but they introduce
more complications for unsophisticated users than they remove \cite{}.

Fortunately, once ADCIRC is built, the resulting binary executables do not need
to be rebuilt to turn different options on or off. This is handled in the
settings file, colloquially known as the fort.15 file.

\subsection{The Geographical Domain, aka, The Mesh}

As mentioned above above in the section on ADCIRC Input Files, the fort.14 file
defines the physical geographical domain of interest. When ADCIRC is involved,
this area will always involve littoral or nearshore areas. Once a mesh is
selected to represent the desired geographical domain, it is a decision that
remains permanent for the duration of the particular forecasting run.

There are development versions of ADCIRC that are able to nest mesh domains and
it is possible to take the state file from one one run and interpolate it onto a
smaller, more focused mesh. But these are used for research purposes and not
used operationally. If this was part of the operational system being discussed
in this paper, then it would only slightly complicate the process by adding an
additional step for introducing another mesh. The applications of this are
interesting, but have so far not been necessary for operational needs.

\subsection{Input Data}

The consideration of acquiring input data, particularly meteorological forcing
data (GFS, NAM, NHC, etc) is an important one. There are many operational
systems that publish forecast data that may be used by ADCIRC at regular
interviews. If someone is automating this part of ADCIRC, then it is safe to
assume they're in some interested in timely or operational results. For example,
give the mesh and the current wind forecast, where will the highest surges
occur? Answering questions like this is in the operational domain.  The main
challenge to automating the acquisition of data consist of:

Over the years, many tools and approaches have been proposed to make it easier;
and general patents from the Navy exist \cite{}. Most people tend to create
their own scripts, but recently there have been attempts to present SaaS
aggregation services that allow one to select which meteorological data to get
and how it should be converted before downloading. While interesting and useful,
a more local approach is required for operational considerations; especially
during an emergency management situation like a hurricane.

\subsection{Updating Model Settings}

After ADCIRC is compiled, input files are used to define the simulations. The
fort.14 is a file that defines the physical geographical domain, e.g., the shape
of the coastline around the areas of interest. From simulation to simulation,
this geographical model remains unchanged.  However, the model parameters are
controlled by the fort.15 file; and this sets a lot of the physical constraints
of the model. It also controls the temporal constraints e.g., the cold start
time, ramp time, and number of days. The start time of the model maps the t0
timestep of the model with a real external date and time in the human world. In
most cases, people who run ADCIRC treat it like a one time run.  In a forecast
mode, the fort.15 must be updated to take into account the continuation from the
previous run. While the start time remains the same, there are parameters that
must be set to tell ADCIRC to, with the benefit of a saved state file, to begin
at a later time.  If one was to do a forecast by hand, the procedure would be:

As will be seen below, Perl offers a uniquely simple way to manage the updating
of this settings file.

\subsection{Preparing a Multiprocessor Execution}

In addition to being an exceptionally accurate physical simulation of storm
surge and the effect of tides in nearshore and littoral environments, one of
ADCIRC's greatest practical strengths is that it scales very efficiently when a
single simulation is spread over multiple processors. In fact, many benchmarking
studies of the code has shown that ADCIRC will scale linearly up to the point
that there are 3-5,000 nodes per CPU or core. \cite{benchmarks}.

In practical terms that means that one may take mesh that has been created with
very high resolution in specific areas (e.g., the southeast
Louisiana/Mississippi Delta) and run them within a bounded amount of time
provided that it is run on enough processors. For example, there is a mesh used
operationally in Texas and Louisiana called the TX2020a that has exactly
4,266,444 nodes and 8,456,596 triangular elements. At 4,000 nodes per CPU, this
helps operational experts justify the use of over 1,000 CPU codes per execution.
And because the speedup has been shown to be linear, it would take the same
amount of time to run this large TX2020a mesh that it would to run a mesh
1/1,000th the size on a single CPU core.Recent efforts out of Notre Dame and
funded by NOAA are producing an extremely large global mesh\cite{}. Notre Dame's
group headed by Professor Johannes Westerink regularly publishes an update mesh.
Each iteration put out by Notre Dame is focused on adding additional resolution
over important areas of the world that require more dense node coverage because
there are very important hydrological dynamics that have been identified to be
occurring in these areas. This mesh is over 12 million nodes, which means that
if run operationally it would efficiently use over 2,500 CPU cores. And many
sites across the United States are capable of providing these resources,
including the Texas Advanced Computing Center
(TACC)\footnote{{https://tacc.utexas.edu/}} and the combined resources of
Louisiana State University's HPC Center\footnote{\url{https://hpc.lsu.edu}} and
thier sister organization, Louisiana Optical Network Inititiative
\footnote{\url{https://loni.org}}.

Preparing ADCIRC input files for parallel execution is relatively simple. When
ADCIRC is compiled, it builds a utility called "adcprep". This utility sole
purpose is to decompose the original, potentially very large mesh and input
files into the specified number of smaller domains and with corresponding
subsets of data for the subdomain, respectively. One may think of the execution
of parallel ADCIRC as the cooperative execution of many mini-ADCIRC domains. The
adcprep utility also tracks which subdomains are adjacent, like puzzle pieces.
And the well established Message Passing Interface standard library (OpenMPI) is
used to facilitate the communication of the values at the shared edges of these
subdomains.

This step is required each time the fort.15 is created for a subsequent run and
for each new set of input files.

\subsection{Executing a Multiprocessor Simulation}

Parallel ADCIRC may be built, prepared, and executed just fine on someone's
many-core laptop or multi-core desktop workstation. But in order to get access
to very large numbers of processors for these large operational meshes, it is
necessary to use the academic or governmental HPC machines. And all of these
environments present time on their large clusters through a time sharing batch
system, generally referred to as the "queue". On a private system, a parallel
ADCIRC execution is initialed with the "mpirun" command.

On a large shared HPC system, one must submit a job to the queue. The "submit
script" contains some important information so the system can validate time
budget to certain projects, what kinds of compute modes should be involved, how
many CPUs are required, etc. Over the years the batch systems have varied based
on vendor. For example, IBM HPC machines use(d?) a system called LoadLeveler.
There is also a system common on old running Linux clusters called PBS. Modern
systems tend to use the open source batch queue system, which works very
similarly to its forerunners, called Slurm.

So, executing parallel ADCIRC on a private system is not a big deal because it's
a single command on a private system, and this can be automated very easily. But
automating the submission of a large ADCIRC simulation on a batch system can get
complicated very quickly, and there are many different situations that must be
considered. And this is all assuming that the automation and submission happens
local to the HPC itself. If the "command and control" of the ADCIRC automation
must be managed on a remote machine; it gets even more complicated.

\subsection{Post Processing}

In the early 2000s when the automation of ADCIRC had not yet been sufficiently
explored, there was an additional step that was essentially the inverse
operation of "adcprep". It is called, unsurprisingly, "adcpost". It's job is to
combine all the output files from all the "little" ADCIRC domain runs, into
global output files. However, a great time saving step was introduced into
ADCIRC around version ZYX \cite{} that allowed for the offloading of the output
writing to a dedicated process among the MPI processes contributing to the
computational needs of the simulation. One or more dedicated writer processes
may be reserved, and when used allows for all of the computational processes to
send their output to the writer process. This writer process than handles the
actual writing of the output files. As a result, the use of adcpost has waned.
It has also been found that there is really no need for more than one of these
dedicated writer processes for the computations of the model to proceed
efficiently.

\subsection{Monitoring Execution Progress}

Monitoring a currently executing simulation is important for detecting any
errors that might be occurring because a silly mistake in the submission process
or a more fundamental issue with the model physics. If one is attempting to
submit a set of tens or even hundreds of ADCIRC simulations (as has been known
to be done), the detecting issues along the way is very helpful in making the
most efficient use of one's own time and any computing time allocated to the
effort.

\subsection{Detection of Simulation End}

\subsection{Output Data Management}

\subsection{Delivery of Results}

\subsection{Continuity During Extended Simulations}

As if this wasn’t complicated enough, ASGS minimizes the amount of ADCIRC
runtime needed by managing state or hot start files. In order to always be
starting at a model state that was generated from where the storm actually has
been, it is essential to play catch up before running the next set up storm
scenarios.  When the NHC issues a forecast, it also issues what’s called a best
track file that gives tells us where the storm has been and what its central
pressures and isotachs were. So when the NHC issues an advisory, what actually
happens is the following:

\subsection{Advanced Capabilities: Model Coupling and Data Assimilation}

The ADCIRC model is for a very specific physical process, i.e., virtually
evolving a body of water on a revolving earth based on both tidal effects and
external forcing. Most often these forcings are in the form of the wind that is
blowing over the water's surface (e.g., the storm surge due to a hurricane
nearing landfall). However, in some geographical areas of the world, considering
the effect of something like waves is important.  In the past, demonstration
projects have been conducted to explore the efficacy of different approaches.
One approach is file based couple; essentially models take turn running, dump
their output to a file, then that output file from one model is used as the
input file of another. This can happen in just one direction (one-way coupling)
or bidirectionally (two-way coupling). Advanced systems using a middleware
broker architecture have also been explored and shown to be very good for
coupling many models into a single system \cite{}.  But in practice, when
it is shown to be truly useful for improving the model results, coupling is done
by integrating models into one another so that they run under the same binary
executable. For example, ADCIRC and SWAN, a wave model, complement one another
so well, that ADCIRC is distributed with a version of SWAN and build a single
"coupled" executable called "ADCSWAN" (ADCIRC+SWAN). \cite{}.

\section{ASGS: ADCIRC Surge Guidance System}

\subsection{Background}

ASGS is a system that grew out of the need to automate ADCIRC forecasts for the
New Orleans District of the US Army Corps of Engineers in the immediate
aftermath of Hurricane Katrina. A system called the Lake Pontchartrain
Forecasting System was built by Estrade, et al.\cite{} at LSU's Center for
Computation and Technology in 2006 and demonstrated at Supercomputing 2006 in
Tampa, Florida. The system utilized LONI's network of IBM P5 supercomputing
clusters that were located around the State of Louisiana.

CCT was funded and tasked by the Corps to produce a system they could use to
detect high predicted surges at 3 different locations of the south shore of Lake
Pontchartrain. The locations were the outlets of 3 large rainwater drainage
canals that carried rain water out of the City of New Orleans during heavy
rainfall. During Hurricane Katrina, the storm surge on the south shore of the
Lake was very high and the open canals actually served to let water into the
City during the storm. Most infamously the 17th Street Canal's eastern wall
collapsed due to the hydrostatic pressure of the surge, and resulted in the
devastating flooding of the New Orleans neighborhood of Lakeview.  The Corps
proposed designed and installed gates at each of these canal locations, that if
needed due to storm surge, could be lowered into place to prevent water
infiltration. However, when lowered these gates would also prevent the drainage
of the rain water to flow out of the City and into the Lake. The solution for
this issue was to add high powered pumps that would lift the drained rainwater
over the gates. They needed a system that would allow the Corps' decision makers
to determine if and when they needed to lower the gates into place and turn on
the pumps.  The forecasting system was triggered off of NHC forecasts. Once a
forecast was issued, the system would prepare an an ensemble of 4 slightly
modified predicted hurricane tracks to be used as the forcing input for ADCIRC's
internal GHAM wind model. Each of the 5 tracks (the consensus NHC track + 4
variations) were intended to run on each of the 5 IBM P5 supercomputers deployed
by LONI.  The demonstration project showcased other aspects of "on demand" HPC
computing. For example the HARC Scheduler\cite{} was used coordinate the
submission of these 5 tracks in order to minimize the turnaround time and
maximize the number of simulations that could run on different machines at any
given time. There was a system monitor that detected when the runs were
finished, and when it was the output files were collected and post processed.
There was some rudimentary attempts to automatically determine a threat level
(out of a simplified 5 point scale); and notifications were part of this
approach, as well. The project was successfully demonstrated at Supercomputing
2006; but it was never used operational.

In 2008, the system was resurrected when the State wanted to begin producing
hurricane surge forecasts for their emergency management teams during the lead
up to what became Hurricane Gustav. In collaboration with the University of
North Carolina (home base of one of ADCIRC's creators, Dr. Rick Luettich); the
scripts used from the original LPFS project were peared down and simplified for
real time operations. The State wanted the most reliable system in place
possible in that time frame, and that consisted of running only on a single HPC
P5 machine, which was located in Baton Rouge. Certain pieces were kept, e.g.,
the detection of when the NHC issued a new advisory was needed. The Perl to
generate an ensemble of alternate hurricane tracks was also need to explore
potential shifts in track, wind intensity, or both. ...

\subsection{Operational Considerations}

Over the years ASGS has grown into a very mature, operationally focused
execution environment with strong support for many aspects outside of its
original purpose of running a continuous ADCIRC forecast. One thing became very
clear each year this system was employed (and it has been ever since with 16
years of continued service to the people of Louisiana and Texas). The thing that
was made clear is that real time emergency operations of a forecast system
required more than just the ability to automate ADCIRC.

During real time emergency operations, the need to be able to adjust many
aspects of the system based on stakeholder demands for scenario information was
found to be very important. The ability to quickly set up a system on a
different machine for a multitude of reasons was also made clear. Minimizing the
cognative load of the operator was also found to be very important; this can
span from eliminating as much type as possible with ready-made tools and scripts
to being able to quickly debug a misbehaving system as quickly as possible.
\subsection{ASGSH - The ASGS Shell Environment}

In 2018, work was undertaken to unify the installation process of ASGS. Until
this time it had been just a bunch of scripts. These scripts worked reliably,
but is was exceedingly difficult to onboard new operators or port to new
systems.

The result was a unified installation script that essentially created an
isolated environment that maintained its own environmental variables. For
example, one of the hardest parts of building ADCIRC from source is that it
requires netCDF libaries for Fortran. This requires the underlying netCDF C
libraries, which in turn requires HDF5 C libraries.  The installation script is
written in Perl, and manages the building of many other packages and makes heavy
use of Perl's natural access to the user's shell environment via the \%ENV hash.
In addition to building upon \%ENV, the script must manage different platforms;
which for the purposes of ASGS constituted specific HPC systems as well as
generally specific Linux distributions. Because of the ASGS installation script
uses a custom system for detecting which platform it is on; this script is aptly
called "guess".

Once the build script is finished running, it dumps a text file that is
basically a shell script that "exports" all the environmental variables it cares
about controlling. For example, it updates PATH, LD\_LIBRARY\_PATH, and
LD\_INCLUDE\_PATH. These variable are essential to building ADCIRC with the local
netCDF library, for example.

ASGS also presents an interactive "shell", which is started by running a
generated shell scripted called "asgsh". When executed, "bash -c" is used to
load this script; and the user is presented with a shell environment that is
still bash, but is loaded with all the proper paths and environment. For the
purposes of the reader's imagination, they may consider this a "REPL" shell. The
advantage of this is that it is isolated from their normal login shell, and
within it they have access to all the standard ASGS utilities and scripts in a
local ./bin directory. The figure below is what the shell environment looks like
when it is first loaded.

(insert figure)

\subsection{Named Storm Profiles}

ASGS now has total control inside of this shell without concern for the overall
system or user's login shell. As such, it supports several features that help
immensely during emergency operations. For example, the ASGS Shell Environment
supports "profile" switching; this allows a general standard operating procedure
of creating a named profile for each named storm of the given year. Within that
profile, the user may modify many different environmental variables that relate
to the parameters used to control the ADCIRC automation, inside of a
configuration file.

\subsection{Support for Many Versions of ADCIRC}

ASGSH also strongly supports the building and management of different versions
of ADCIRC. It is very similar to what perlbrew provides for Perl. In addition to
this, there is an ADCIRC build wizard that allows a user to select from many
supported versions (and deviations from versions, called flavors). This wizard
is built upon git and a patch system that was developed to patch the upstream
ADCIRC code. The figure below is a screen capture of what the ADCIRC build
wizard looks like at the time of the writing of this paper. Behind most options
for ADCIRC version, is a set of patch files that are maintained by the ASGS
project. The patches generally adjust the Makefile or work/cmplrflags.mk file;
but have been used in the past to adjust the function Fortran 90 source code.
The true power of this system has yet to be realized, but offers a tremendous
opportunity for customizing ADCIRC behavior and user interface to support many
different use cases.

(insert figure)

\subsection{Support for Many Platforms}

Extending ASGSH (or porting) ASGSH to a different platforms is also supported,
because this is an absolute requirement to remain operationally flexible. The
major HPC systems that are utilized during the hurricane season have their own
specific support because each one is slightly different. Most of these systems
run some version of Redhat Enterprise Linux (RHEL), but each one is set up
differently - even within the same HPC organizations. The result of the need to
manage these differences has resulted in a very flexible system.  One of the
differences among the HPC systems is the compiler suite that is offered. The
Intel Compiler suite is used almost universally across all institutional HPC
environments, but there are differences among the versions of this suite that
necessitate they be treated specially. For example, porting ASGS to TACC's most
recent HPC addition, Stampede3 \cite{}, required that ASGS deepen its support to
add an additional class of compilers. It now supports the
GCC/GFORTRAN\footnote{version =< 9...}, the traditional Intel Compiler suite
(which provided icc, ifort, mpif90), and now the new Intel OneAPI Compiler suite
that provides a new set of LLVM based compilers, for C this is the icx compiler
and for Fortran, the ifx compiler. Intel has also introduced another wrinkle
regarding the MPI compiler.For many years Intel cooperated with the psuedo
standard of naming the MPI Fortran compiler, "mpif90". In recent years, it has
moved towards favoring this compiler (wrapper) named "mpiifort"). It remains to
be seen if they introduce something a "mpifx" compiler, but their recent history
suggests this will be, or already is, the case.

\subsection{Build Around ADCIRC Automation}

Generating Alternate Storm Track Scenarios

One of the major concepts that LPFS demonstrated was the efficacy of running
multiple alternate storms. The reason for this is that, due to the complexity of
coastline geography, drastic changes in expected storm surge may happen based on
real time shifts in the storms actual path or in its actual strength. For this
reason, ASGS runs not only the official forecast (also called the consensus
track), but also some tracks that have been perturbed in such as way that they
may veer right or left to some degree; or the wind speed may be increased or
decreased.  During a storm event, the scenario package may be adjusted based on
the information needs of key stakeholders at the state and federal level. For
example, given the consensus track; they may wish to see the surge forecast in
the event that the storm shifted to the left 50\%. They may also wish to see
what would happen if the storm intensified at some point before landfall, which
in the model would mean that the wind speeds would be increased by some
percentage. In ASGS, storm scenarios may be adjusted for both the veer and the
change in wind intensity.

\section{Operational Support}

The techniques and tooling built into ASGS have their roots in operational
necessity. One of the very first continuous forecasting systems developed around
ADCIRC was in support of Operation Iraqi Freedom\cite{}. The system itself
utilized shell scripts that had been used originally for research runs of ADCIRC
on NAVO's Major Shared Resources Center HPC resources, and were adapted to run
in a classified ARL HPC environment. The scripts downloaded timely
meteorological forecast data, converted this data for fort.22 file formats by
interpolating this data onto the ADCIRC mesh, prepared input files for a
multiprocessor ADCIRC execution, and submitted this job to the batch system of
the HPC ARL machine. When it was detected that the simulation had run to
completion, the results were downloaded to a local machine, and plots were
generated using Matlab scripts. These graphical plots were then uploaded to a
secure FTP location and made available to decision makers.

Even though this system was cobbled together with shell scripts and a very
cumbersome Matlab environment that was difficult to incorporate into an
automated workflow, it demonstrated the efficacy of the approach. The system
continued to evolve, and was eventually using an XML based framework that was
used to describe a series of remote and local interactions using ssh. In many
ways, it resembled systems that are now used by system administrators for
provisioning and managing large amounts of systems, e.g., Puppet, Chef, Ansible.
It was called Securerun and it is now on github\cite{Securerun}. A Patent
application was awarded for this technology, but it was never approached with
its own USPTO Patent\cite{SecurerunUSPTO}. However, the system that was developed to fetch and
process the meteorological data was awarded a USPTO Patent.  This experience
with the OIF system directly impacted the architecture and approach of the LPFS
system that was developed in 2006 at LSU's Center for Computation and
Technology. However, neither of these systems were considered "real time" or had
the kind of delivery constraints that ASGS now supports.  ASGS is now a critical
tool for emergency managers in Louisiana, Texas, and among several federal
agencies for making important decisions. It is not the only tool used to help
aid in the decisions, but it is an important one.  Examples of decisions that
are made based on the real time surge forecasts of ASGS include:

As an example of the impact that ASGS has on emergency operations nationwide,
ADCIRC was the winner ofDHS Science and TechnologyImpact Awards in 2010 and
2012\cite{}.

\subsection{Operator Friendly Feature}

The developers of ASGS (Fleming and Estrade), are serve as operators themselves
during tropical storms. And as a result they have key insight into the kinds of
features that make their lives easier and less prone to mistakes, particularly
after being up for many hours. When a storm is threatening the US coast, the
ASGS Operations Team is on call 24/7 for the duration of the event. This aspect
of the system, the human operator, is not something that can replaced with a
program. But they can be greatly assisted by the right type of programs and
tools.

It starts with ASGS' robust approach to reliable installations of the tools and
supporting libraries. It is very easy to install ASGS on all of the big HPC
systems that are required to run ADCIRC on large, operational meshes.

Once ASGS is built, the operator starts the ASGS Shell Environment, which is
outfitted with many special purpose tools and commands that an operator will
need to use when setting up a forecast. There is a named profile system that
allows operators to save and recall previous set ups, commands to clone these
set ups for rapidly duplicating a known good setup, and recovering a profile
from an orphaned configuration file. There are tools that are used to verify
that the installation has everything it needs to run the Fortran, Bash, and Perl
utilities that are so critical.

The commands have been so streamlined that once ASGS is built, a trained
operator can have a forecast running a custom set of storm scenarios (scenario
package) in mere minutes. And a single ASGS installation can handle an arbitrary
number of simultaneous forecasting runs, limited only by the capacity of the
number of CPU cores the system has made available for this work (which usually
numbers in the thousands).

The developers of ASGS are always looking for ways to streamline common tasks
and situations that take a non-trivial amount of time to recover from. A
relatively recent capability allows an operator to make a back up of the
installed system and place it safely in another directory or on a remote system,
ready to provide a way to recover a known good systems. This capability was made
as a result of some HPC center's having a fairly aggressive data purge policy.
This prevents operators from keeping a known working installation intact for
more than 60 days, so no staging is really possible more than 30 days out.

Because the operators are often making adjustments at all hours of the day, a
Perl script was developed that would statically evaluate an ASGS configuration
file. It is easy to mistype some parts of the configuration file. It is called,
asgs-lint. And another new tool that is available for experimental use is
asgs-mon, and standalone monitoring program that is able to run a number of Bash
and Perl scripts periodically to evaluate the health of the ASGS forecast
currently underway. I can alert an operator if there are issues detected and it
may be used to perform various housekeeping tasks. This capability is one that
has been sorely needed, and it is possible because Bash and Perl are so reliable
and predictable.

For new operators, or anyone who wishes to play along at home, there is a public
checklist for onboarding new operators. There is also an Operators
Cheatsheet\cite{} that as designed for both new and experienced operators to
reference when in the thick of things. There are many shortcut commands and
tools for navigating the ASGS directory hierarchy, many are specific to the
current forecast. For example, when typed in the ASGS Shell prompt, "rd" is an
alias to "goto rundir"; this will change their current working directory to be
the one that is supposed to be where all the ADCIRC runs are happening for each
forecast that is issued. It was created because operators kept having to look at
the contents of a "state file" for this information. There are many other such
commands that eliminates the need for the operator to have to hunt for
information or have a hard time navigating important directories.

\subsection{Redundancy and Battle Rhythm}

An important aspect to executing with confidence in a high pressure, real time
emergency is that the ASGS Ops team has learned to hedge its bets and duplicate
its most important runs on different HPC systems - ideally separated by some
large physical distance, e.g., LSU HPC/LONI and UT's TACC. It is possible to
grab the output of a set of forecasts operating in one location (say, Louisiana)
and hot start it another location (e.g., Texas). This is crucially important
when a storm is heading towards a specific area. For example, Baton Rouge is
usually not spared from hurricanes even though it's some more miles inland that
New Orleans. But obviously, the State of Louisiana would want the forecasts for
the storm about to impact it to be done on State resources.

But, in 2008 for Hurricane Gustav; it is common for such storms that directly
impact Baton Rouge to knock out its power. This didn't happen back then, but it
regularly does no; but the solution would be to run a simultaneous forecast in
Texas so that when the power in Baton Rouge goes out and the LSU machines go
down, the run goes on. The hotstart handoff feature of ASGS was designed to
easily faciliate this kind of operatinal coordination.

\subsection{Technology Decisions}

The software stack is outlined in much more detail in the following section, but
the decision regarding what technologies to use center around a few important
aspects. The developers of ASGS have been working around Linux for a long time,
over 20 years. And they settled on using certain tools and learned to use them
very well. These tools, in particular Bash and Perl, also work exceptionally
well and allow for the creation of extremely long lived scripts and utilities.
Using stable and well support tools has contributed directly to the stability
and robustness of ASGS, and this really is apparent during when a hurricane is
bearing down on the US coastline; and every minute counts but also a mistake can
mean that emergency managers do not get the information that they need at a
critical time.

\section{The Software Stack}

More details about the operational considerations when deciding the software
technology used to build ASGS and support ADCIRC automation are covered in the
\section{Operational Support} section. This section focuses on the technology
that underlies everything. It suffices to say here that the technology decisions
are based on the following criteria:

It is safe to assume that all important systems used when ASGS are being used
operationally during an active Atlantic hurricane are running some variant of
Linux. It is true that over the years, as HPC system vendors came and went, they
provided their own operating system environments. During the lifespan of ASGS,
the machines that were used operationally were either IBM P5s running AIX or
they were basic beowulf Linux clusters running in most cases, Red Hat Enterprise
Linux. The part of these systems that made them different than a typical
commodity cluster is that they had access to extremely large amounts of disk
storage and that they used the fastest possible network topologies, which even
today are optimized for message passing (MPI) based parallelism by allowing
direct internode memory access that use a special protocol to read or write
messages directly to the system memory (e.g., Infiniband supports this type of
outreach via their special DMA card and protocol "verb" technology).

\subsection{Linux}

Fortunately, the userlevel interfaces of these libraries are sufficiently mature
to allow  a C and Fortran codes to compile virtually unaltered when using any of
the MPI libraries. The other, more fortunate aspect, is that the userland
environment was already well mature and standardized on all of these machines.
And while each system has its idiosyncrasies, they are all fundamentally
architected in the same way.  Over the years, the stability of the GNU userland
tool chain has had a profound effect in a vast number of areas due to it's
standardizations and predictability. This allows ASGS to fully expect and take
advantage of standard Unix utilities; such as sed, grep, find, and awk. Access
to these standard utilities is essential to the core of what ASGS does, ADCIRC
automation on the large systems required for operational meshes, and to minimize
the developer load in maintaining portability.

\subsection{Fortran 90}

The ADCIRC model as originally written in Fortran 77. Over the years, it has
been brought up to use Fortran 90. This is a solid choice for what it does; more
importantly the subject domain experts who wrote the original ADCIRC source code
and those who continue to maintain and add to it today, are more experts in
Fortran than any other language. It was written for scientists, and the
scientists who advanced ADCIRC from their research leverage it quite well. It is
also well supported by GNU's GCC suite through Gfortran. This suite also
continues to live on and bear much fruit. There are also no questions regarding
the ability for commercial compilers, such as those provided for by Intel, IBM,
Portland Group, NAG, etc; to build this code.

There are also occasional situations where ASGS needs a capability, and because
the knowledge and will to use it exist among the ASGS developers; Fortran is the
preferred choice for creating utilities that much deal with input or output
data. Obviously, Fortran falls short in areas for which it is not intended. Text
processing and system process management are two areas in which ASGS would not
attempt to use Fortran. But, nobody is going to rewrite ADCIRC in C, C++, or
something like Rust ever. This is a bold claim, but a safe one. So Fortran 90 is
a permanent member of the ASGS technology stack.

\subsection{Bash}

Because of the assurances that exist regarding the availability of standard GNU
userland tools and the widespread adoption of Linux on the big systems ASGS
requires during realtime operations, Bash is also a permanent member of the
technology stack.

As mentioned previously, ASGS began from the ashes of a project that
demonstrated the benefits of an ensemble based approach. Nearly anyone in the
last 25 years who has attempted to automate any aspect of ADCIRC has utlized
shell scripts to orchestrate all of the steps needed. It is the obvious choice
to glue together standard Unix utilities frequently used and the tools that have
been built using Fortran 90 and Perl.

It is a Bash script that is the primary driver for the forecasting system
itself, which implements all of the automation steps required for ADCIRC
forecasting that were outlined above. This script is called \textit{asgs\_main.sh}.
Another strength of shell scripts is that it has direct access to the
environment, and this is the only reason that the ASGS Shell Environment exists
and is able to provide so many benefits.

The reliance on shell scripting to orchestrate everything also allows the entire
system to be immediately accessible and field servicable during a real time
storm event. It is the basis for allowing the team to mitigate a great number of
service disruptions that are outside of their control. It may be surprising for
those familiar with ASGS or the team behind it that Perl is not the main driver
of the system or enabling technology behind the custom interactive shell
environment; but it is true. Perl provides a strong basis for developing very
powerful and complex tools, but it within the Bash scripts that these Perl
utilities are generally called just like any other shell command or userland
utility.

Specific qualities and capabilities of Bash that are very important include:

As a final note, the actual Bash shell (bash) that is used is the one provided
for on the system itself. This note is to contract how Perl is handled, as
describe in the preceding section.

\subsection{Perl}

Perl is a critical piece of ASGS, but is used to create very targeted tools. It
is also used to create tool systems, such as the installation script
(asgs-brew.pl) and the ASGS monitor (asgs-mon).

It is important to note that, unlike the system "bash" shell; ASGS does not rely
on the system perl. This is generally not because the system "perl" is usually
old, because Perl scripts are in many cases exceedingly backward compatible. But
it is because the ASGSH environment relies on many Perl modules that would be a
pain to deal with from system to system. So that the HPC system administrators
do not have to get involved, ASGS build its own perl using the defacto tool for
creating Perl virtual environments, perlbew. One exception to "not using system
perl" rule is that asgs-brew.pl must be written in a way that assumes only the
most basic Perl interpreter. It is considered a bootstrap script, but this is
not a hard constraint. asgs-brew.pl is written so that it can be run as far back
as perl 5.8.8 - maybe further back. That is very widely deployed version from
over 20 years ago.

Perlbrew allows ASGS to build up a very recent version of perl and install all
the modules it needs directly to CPAN. ASGS does have a Perl library it
distributes in the PERL/ directory; but it takes full advantage of the ability
to download modules it needs to do amazing things.

What are these amazing things?

\subsection{The Technology Trash Heap}

\section{Other Uses for ASGS}

The software ecosystem that has been built around ADCIRC and to support
operational use through ASGS has bore much good fruit about making it useful in
situations that are not real time and not affected by delivery timeframes.

\subsection{Operational Research \& Development}

ASGS, with it's support to build and load for use many different versions of
ADCIRC, has resulted in the ASGS Shell environment being really super handy for
working with ADCIRC manually.

The patch and ADCIRC version management system for building and loading
different versions of ADCIRC a breeze. The patch system is so simple, additional
patchsets may be created very easily just by copying an existing one. This is an
ideal environment for experimenting with ADCIRC. There are a lot of tooling
ideas around these activities, but there has not yet been enough demands to
develop this more fully. The future is bright, though. A lot of people are
looking at ASGS to support their non-real time work.

\subsection{ADCIRC Only Builder, On Windows}

The developers and main operators have recently released a site for getting
commercial ADCIRC support. It is called ADCIRC Live
(c)\footnote{https://adcirc.live}. One of the first available tools is a
single script that is able to build ADCIRC directly on many versions of Linux,
including those hosted on Windows under WSL2. Being able to reliably build and
use ADCIRC locally under Windows has been something that many people can use.
The big deal here is that it can provide for a way to manage different ADCIRC
versions and load them without the user being inside of the interactive ASGS
Shell. While effective, the shell is primarily used for interactive real time
operations. But it is also a powerful environment for doing anything with
ADCIRC, because it has to be for real time operations.

\subsection{ADCIRC Focused SaaS and API Tools} Related to the ADCIRC build in
the previous section, the same client may be used for making client calls to the
ADCIRC SaaS API. It currently supports API calls that make it very easy to
visualized ADCIRC data locally in Paraview.

\section{Conclusion}

\section{Acknowledgements} (need to fill this in, ASGS was funded along way by
several state and federal agencies; and supported by several HPC centers and
researchers at Notre Dame, LKSU, UNC, UT, North Carolina State, and the
University of Oklahoma(I think))

\break
\bibliography{references}
\end{document}
