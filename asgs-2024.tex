\documentclass{article}
\usepackage{blindtext}
\usepackage[a4paper, total={6in, 8in}]{geometry}
\bibliographystyle{plain}

\title{Building A Storm Surge Forecasting System That Saves Lives}

\author{
Brett Estrade \\
\large\textit{brett.estrade@coastalcomputingservices.net}
\\
Jason G. Fleming, PhD. \\
\large\textit{jason.fleming@seahorsecoast.com}
}

\begin{document}
\maketitle

\begin{abstract}
The ADCIRC Surge Guidance System is a portable real-time operational storm surge
forecasting framework, forged over 15 Atlantic hurricane seasons, that is used
to deliver critical information to emergency managers on federal, state, and
local levels in Louisiana, Texas, and North Carolina and within groups such
FEMA, NOAA, and DHS. Over the years it has saved millions of dollars in time,
property, and emergency assets. It has also likely saved many lives. This paper
discusses its humble origins from a collaboration between LSU and UNC in the
early wake of Hurricane Katrina (2005), to its first real test during Hurricane
Gustav (2008) and successes over the years. Much of this paper discusses the
technical aspects of the system, how the user experience has been tailored for
real-time operations, and the technical decisions that have been made leading
directly to its success as a robust and adaptable framework.  Most relevant 'for
the intended audience is how using Perl, Bash shell scripting, and standard Unix
tools has made all of this possible.
\end{abstract}

\section{Introduction}

It is hard to believe that in 2001 when weather forecasting models were well
know and used across society, that the the idea of providing such forecasting
litoral, or nearshore zones was the realm of specialized oceangraphers and the
military. But it is true, that until the infamous Hurricane Katrina in 2005 that
that nobody outside this small cadre of specialists ran these kinds of models or
cared much.

As it came to pass, Katrina exposed the crucial need for providing autmated
storm surge forcasting. And wider society was reminded that it is the water that
kills the most people during hurricanes, not the wind. Far more than a slow
moving, foreboding and wet tornado, a hurricane is a wholly different kind of
threat. Along with its many inches of rain and harsh winds, can come a wall of
that may in some places exceed 30 feet. And this was the case with Katrina.

The mathematics behind such modeling capabilities had existed for a long time (CITE),
and infact was pioneered by the Dutch; who themselves had been victims of grave
flooding events over the centuries. It was not until the late 1980s that a
model was developed that would revolutionize the ability to accurately model
the way that water behaved near the shorelines.

\subsection{ADCIRC - The Advanced Circulation Model}

... about ADCIRC \cite{luettich1992adcirc} ...

The ADCIRC model is the main \textit{kernel} around which the forecasting system
ins build. It automates all aspects of setting up the model, running it (usually
in on a parallel Linux cluster), processing the results, and send the results to
a geographical information system for viewing on the web.

The remainder of the paper will describe the automation built around this model.

\section{ASGS: The Little ADCIRC Engine That Could ...}

About ASGS ... \cite{fleming08}

INSERT SYSTEM DIAGRAM HERE

\section{Managing Input Data}

\subsection{The Unstructured Mesh}

... fort.14

\subsection{North American Meteorological Forecasts (NAM)}

ASGS may be run in a \textit{daily} forecasting mode where it watches for
available NAM forecast data. This data is typically made available 4 times in a
24 hour period at: 00Z, 06Z, 12Z, and 18Z.

This data consists of ..

ASGS uses a Perl script to create a meteorological wind forcing file, commonly
referred to as a \textit{fort.22}. The data conversion process involves
interpolated the NAM gridded data values onto the unstructured nodes of the
unstructured mesh that ADCIRC uses to represent complex coastline geometries.

NAM data is also used when the NHC identifies areas of interest, but are not yet
well formed enough to begin issue advisories. ASGS is able to shift from NAM mode
to NHC mode once official advisories begin. These advisories are short text file
and do not contain explicit forecast data, so here again Perl is used.

\subsection{National Hurricane Center Forecasts}

The advisories that are issued but the NHC contain forecasting about the future
projected path of the center of pressure of the system. In addition to this, it
contains a set of date that describes the forasted path of the system and lines
of similar pressures (VERIFY).

A Perl script parses this information out of the NHC forecast advisory, and converts
it into a format that may be used by ADCIRC to generate a wind field gradient.
This wind field is then mapped internally inside of ADCIRC to the mesh, and used
to drive the model state as if it were provided explicit wind information. The
internal wind generation is done using what's called the \textit{Generalized Asymmetric
Holland Model}, or just GAHM.

\section{Generating Alternate Storm Track Scenarios}

One of the major concepts that LPFS demonstrated was the efficacy of running multiple
alternate storms. The reason for this is that, due to the complexity of coastline
geography, drastic changes in expected storm surge may happen based on real time
shifts in the storms actual path or in its actual strength. For this reason, ASGS
runs not only the official forecast (also called the \textit{consensus} track), but
also some tracks that have been perturbed in such as way that they may \textit{veer}
right or left to some degree; or the wind speed may be increased or decreased.

During a storm event, the \textit{scenario package} may be adjusted based on the
information needs of key stakeholders at the state and federal level. For example,
given the consensus track; they may wish to see the surge forecast in the event that
the storm shifted to the left 50\%. They may also wish to see what would happen if the
storm intensified at some point before landfall, which in the model would mean that
the wind speeds would be increased by some percentage. In ASGS, storm scenarios may
be adjusted for both the \textit{veer} and the change in wind \textit{intensity}.

\section{Updating Model Settings}

After ADCIRC is compiled, input files are used to define the simulations. The fort.14
is a file that defines the physical geographical domain, e.g., the shape of the coastline
around the areas of interest. From simulation to simulation, this geographical model
remains unchanged.

However, the model parameters are controlled by the fort.15 file; and this sets a lot
of the physical constraints of the model. It also controls the temporal constraints
e.g., the \textit{cold start} time, \textit{ramp time}, and number of days. The start
time of the model maps the t0 timestep of the model with a real external date and time
in the human world. In most cases, people who run ADCIRC treat it like a one time run.

In a forecast mode, the fort.15 must be updated to take into account the continuation
from the previous run. While the start time remains the same, there are parameters that
must be set to tell ADCIRC to, with the benefit of a \textit{saved state} file, to begin
at a later time.

If one was to do a forecast by hand, the procedure would be:

1. make initial run with input data, save model state at the end

2. update meteorological forcing input files

3. update the fort.15 to specify a \textit{hot start} and the simulation timestep in
which to begin

...


\section{Executing the Model}

..queue systems, automatic detection of finishing, etc ...

\section{Post Processing}

\section{Output Data Management}

\subsection{Managing Remote Data Targets}

\subsection{Offsite Visualization Pipeline}

\subsection{Notifications}

\section{Continuity During the Storm}

As if this wasn't complicated enough, ASGS minimizes the amount of ADCIRC runtime needed
by managing state or \textit{hot start} files. In order to always be starting at a model
state that was generated from where the storm actually has been, it is essential to play
\textit{catch up} before running the next set up storm scenarios.

When the NHC issues a forecast, it also issues what's called a \textit{best track} file
that gives tells us where the storm has been and what its central pressures and isotachs
were. So when the NHC issues an advisory, what actually happens is the following:

\subsection{The Hindcast}

\subsection{The Nowcast}

.. catch up

\subsection{The Forecast}

....

\section{Operational Considerations}

\subsection{System Portability}

\subsection{System Configurability}

\subsection{Redundancy}

...multiple sites, ...

\subsection{Maximization of Operator Efficiency}

\subsection{Minimization of Cognative Load}

\subsection{Field Servicability}

\subsection{Teardown and Rebuild vs Troubleshooting}

\section{The ASGS Shell Environment}

\subsection{How It Works}

\subsection{The ADCIRC Builder}

\subsection{ADCIRC Live ...}

\section{ASGS System Monitoring}

\section{Team Communication and Battle Rhythm}

\section{Technology Decisions}

\subsection{The Software Stack}

based on portability, easy of understanding, modifying in real time, field
servicing, track record of reliabilty, etc, etco

.. history of forecasting systems I've built and the common threads (bash,
perl, fortran ..)

\subsection{The Hardware Systems}

cloud versus bare metal - why ...

\section{Bash}

Bash is the engine ...

\section{Perl}

Perl is the super charger nitro booser ...

\section{The Technology Graveyard}

... python, matlab

\section{Research \& Development}

\textbf{Weather::NHC::TropicalCyclone} ....

\textbf{OpenMP::Simple} ....

... experimenting mesh partitioning schemes

\break
\bibliography{references}
\end{document}
